\documentclass{beamer}
\usepackage{alltt}
\usepackage{listings}
\usepackage{tikz}
\usepackage{graphicx}
\usepackage{color}

\usetikzlibrary{arrows,shapes}

\usetheme{Warsaw}

\hypersetup{
    colorlinks=true,
    linkcolor=red,
    citecolor=green,
    filecolor=magenta,
    urlcolor=blue
}

\title{ICPC Sessions \\
    OR \\
    How to Solve Problems
}
\author{Sebastian Claici \\
    \texttt{sebastianclaici@gmail.com}
}

%% Styling for graphs
%% \tikzstyle{vertex}=[circle,fill=black!25,minimum size=20pt,inner sep=0pt]
%% \tikzstyle{edge} = [draw,thick,-]
%% \tikzstyle{weight} = [font=\small]
%% \tikzstyle{selected edge} = [draw,line,width=5pt,-,red!50]
%% \tikzstyle{ignored edge} = [draw,line,width=5pt,-,black!20]

\begin{document}
\maketitle

\begin{frame}
    \frametitle{A Note on Programming Languages}
    
    Here's what you can use in the ACM ICPC:

    \begin{itemize}
        \item C
        \item \only<1>{C++} \only<2>{\textcolor{red}{\textbf{C++}}}
        \item \only<1>{Java} \only<2>{\textcolor{red}{\textbf{Java}}}
        \item Pascal
    \end{itemize}
\end{frame}

\begin{frame}
    \frametitle{A Note on Programming Languages}

    Some useful tools:

    \begin{itemize}
            \pause
        \item C++
            \begin{itemize}
                    \pause
                \item The GNU toolchain - GCC and GDB
                    \pause
                \item Your favourite text editor (*cough* Vim *cough*) by heart.
                    \pause
                \item For Windows: Code::Blocks, FAR Manager, Visual Studio
            \end{itemize}
            \pause
        \item Java
            \begin{itemize}
                    \pause
                \item Eclipse
                    \pause
                \item IntelliJ IDEA
                    \pause
                \item NetBeans
            \end{itemize}
    \end{itemize}
\end{frame}

\begin{frame}
    \vspace*{\fill}
    \begingroup
    \centering
    \begin{center}
        \huge \textbf{Complete Search}
    \end{center}
    \endgroup
    \vspace*{\fill}
\end{frame}

\begin{frame}
    \frametitle{Strategies to Solve}
    \begin{itemize}
            \pause
        \item Keep it simple.
            \pause
        \item Every problem can be solved through complete search.
            \pause
        \item The trick is to recognise those that would work in time.
            \pause
        \item A complete search algorithm should be the first one you think about.
            \pause
        \item A good rule of thumb:

            \pause
        \vspace{5mm}
        \emph{If it's less than 100 million operations, it will work in time.}
    \end{itemize}
\end{frame}

\begin{frame}
    \frametitle{Example}
    \Large In how many ways can you place 8 queens on a chessboard so that they do not attack each other?

\end{frame}

\begin{frame}
    \frametitle{Example}
    Here's one way:

    \begin{center}
        \includegraphics[scale=0.4]{board.jpg}
    \end{center}
\end{frame}

\begin{frame}
    \frametitle{Ideas}

    We could go over all possible ways to place the 8 queens, and check at the end whether it's valid.
    
    \begin{itemize}
            \pause
        \item 8 places on each row, with 8 rows for a total of $8^8$ possibilities.
            \pause
        \item Would work, but it's too slow ($8^8 = 16777216$)
            \pause
    \end{itemize}
        
    \pause
    We can do better \pause - check if the current configuration isn't valid.

    \begin{itemize}
            \pause
        \item At every point we have at most one queen placed on every row and column.
            \pause
        \item Reduces the search space to $8! = 40320$.
    \end{itemize}
\end{frame}

\end{document}
