\documentclass{beamer}
\usepackage{alltt}
\usepackage{listings}
\usepackage{tikz}
\usepackage{graphicx}

\usetikzlibrary{arrows,shapes}

\usetheme{Warsaw}

\title{ICPC Sessions \\
    OR \\
    How to Solve Problems
}
\author{Sebastian Claici \\
    \texttt{sebastianclaici@gmail.com}
}

%% Styling for graphs
%% \tikzstyle{vertex}=[circle,fill=black!25,minimum size=20pt,inner sep=0pt]
%% \tikzstyle{edge} = [draw,thick,-]
%% \tikzstyle{weight} = [font=\small]
%% \tikzstyle{selected edge} = [draw,line,width=5pt,-,red!50]
%% \tikzstyle{ignored edge} = [draw,line,width=5pt,-,black!20]

\begin{document}
\maketitle

\begin{frame}
    \frametitle{What is the ACM ICPC?}
    \begin{itemize}
        \item The ACM International Collegiate Programming Contest!
            \pause
        \item Team based (teams of 3)
            \pause
        \item The most prestigious global programming competition (since 1977)!
    \end{itemize}
\end{frame}

\begin{frame}
    \frametitle{This is how it goes...}
    \begin{itemize}
        \item Regionals and Finals
            \pause
        \item We are part of the Northwestern European region
            \pause
	\item Top 3 teams will qualify for the Finals in St. Petersburg
            \pause
        \item This means we will make Germany, Belgium, the Netherlands and others cry!
    \end{itemize}
\end{frame}

begin{frame}
    \frametitle{What is the ACM ICPC?}
    \begin{itemize}
        \item The ACM International Collegiate Programming Contest!
            \pause
        \item Team based (teams of 3)
            \pause
        \item The most prestigious global programming competition (since 1977)!
    \end{itemize}
\end{frame}

\begin{frame}
    \frametitle{This is how it goes...}
    \begin{itemize}
        \item Regionals and Finals
            \pause
        \item We are part of the Northwestern European region
            \pause
	\item Top 3 teams will qualify for the Finals in St. Petersburg
            \pause
        \item This means we will make Germany, Belgium, the Netherlands and others cry!
    \end{itemize}
\end{frame}





\begin{frame}
    \frametitle{What is this about?}
    \begin{itemize}
            \pause
        \item Algorithms; more than you did in COMP1009!
            \pause
        \item Using algorithms to solve problems!
            \pause
        \item Beating Cambridge (and everyone else)!
    \end{itemize}
\end{frame}

\begin{frame}
    \frametitle{Recommended Book(s)}
    \begin{columns}[t]
        \begin{column}[T]{5cm}
            \begin{itemize}
                \item \textbf{Introduction to Algorithms}
                    \begin{itemize}
                        \item Thomas Cormen,\\ Charles Leiserson,\\ Ronald Riverst,\\ Clifford Stein
                    \end{itemize}
                \item \textbf{Algorithms in C/C++/Java/}
                    \begin{itemize}
                        \item Robert Sedgewick
                    \end{itemize}
            \end{itemize}
        \end{column}
        \begin{column}[T]{5cm}
            \includegraphics[scale=0.5]{book.jpg}
        \end{column}
    \end{columns}
\end{frame}

\begin{frame}[fragile]
    \frametitle{Binary Search}
    \lstset{language=C}
    \begin{lstlisting}
    int binary_search(int *array, int n, int x)
    {
        int lo = 0, hi = n - 1;
        while (lo < hi) {
            int mid = lo + (hi - lo) / 2;
            if (array[mid] < x)
            lo = mid + 1;
            else hi = mid;
        }

        if (lo == hi && array[lo] == x)
        return lo;
        return -1;
    }
    \end{lstlisting}
\end{frame}

\begin{frame}   
    \vspace*{\fill}
    \begingroup
    \centering
    \begin{center}
        \huge \textbf{Ad-hoc Problems}
    \end{center}
\endgroup
\vspace*{\fill}
\end{frame}

\begin{frame}
    \frametitle{Characteristics}
    \begin{itemize}
            \pause
        \item Most of them are very easy; some of them are very very hard
            \pause
        \item Don't require any special knowledge of algorithms
            \pause
        \item There is always at least one in competitions
    \end{itemize}
\end{frame}

\end{document}
