\documentclass[svgnames,dvipsnames,usenames]{beamer}
\usepackage{alltt}
\usepackage{listings}
\usepackage{tikz}
\usepackage{graphicx}
\usepackage{color}
\usepackage{xcolor}
\usepackage{caption}
\usepackage{subcaption}

\usetikzlibrary{arrows,shapes}

\usetheme{Warsaw}

\hypersetup{
    colorlinks=true,
    linkcolor=blue,
    urlcolor=blue
}

\title{ICPC Sessions \\
    OR \\
    How to Solve Problems
}
\author{Sebastian Claici \\
    \texttt{sebastianclaici@gmail.com}
}

%% Styling for graphs
%% \tikzstyle{vertex}=[circle,fill=black!25,minimum size=20pt,inner sep=0pt]
%% \tikzstyle{edge} = [draw,thick,-]
%% \tikzstyle{weight} = [font=\small]
%% \tikzstyle{selected edge} = [draw,line,width=5pt,-,red!50]
%% \tikzstyle{ignored edge} = [draw,line,width=5pt,-,black!20]

\begin{document}
\maketitle

\begin{frame}
    \frametitle{What is the ACM ICPC?}
    \begin{itemize}
            \pause
        \item The ACM International Collegiate Programming Contest!
            \pause
        \item Team based (teams of 3)
            \pause
        \item The most prestigious global programming competition (since 1977)!
    \end{itemize}
\end{frame}

\begin{frame}
    \frametitle{This is how it goes...}
    \begin{itemize}
            \pause
        \item Regionals and Finals
            \pause
        \item We are part of the Northwestern European region
            \pause
	\item Top 3 teams will qualify for the Finals in St. Petersburg
            \pause
        \item This means we will make Germany, Belgium, the Netherlands and others cry!
    \end{itemize}
\end{frame}

\begin{frame}
    \frametitle{What is this about?}
    \begin{itemize}
            \pause
        \item Algorithms; more than you did in previous years (if any)!
            \pause
        \item Actually using algorithms to solve problems!
            \pause
        \item Beating Cambridge (and everyone else)!
    \end{itemize}
\end{frame}

\begin{frame}
    \frametitle{Why study this?}
    \begin{flushleft}
        \pause
        \includegraphics[scale=0.2]{google.jpg}
    \end{flushleft}
    \begin{flushright}
        \pause
        \includegraphics[scale=0.3]{facebook.jpg}
    \end{flushright}
    \begin{flushleft}
        \pause
        \includegraphics[width=.4\linewidth]{microsoft.jpg}
    \end{flushleft}
\end{frame}

\begin{frame}
    \frametitle{How to Prepare}
    \begin{itemize}
            \pause
        \item Beginners:
            \begin{itemize}
                    \pause
                \item \href{http://train.usaco.org/usacogate}{USACO}, \href{http://uva.onlinejudge.org/}{UVa}, \href{http://projecteuler.com}{Project Euler} (latter not recommended)
            \end{itemize}
            \pause
        \item Intermediate:
            \begin{itemize}
                    \pause
                \item \url{http://poj.org}, \url{http://acm.tju.ed.cn}, \url{http://acm.sgu.ru}, \url{http://acm.timus.ru} 
            \end{itemize}
            \pause
        \item Everyone:
            \begin{itemize}
                    \pause
                \item \href{http://topcoder.com/tc}{Topcoder} and \href{http://codeforces.com}{Codeforces} - make accounts on both
            \end{itemize}
            \pause
        \item Romanians:
            \begin{itemize}
                    \pause
                \item \href{http://infoarena.ro}{Infoarena}
            \end{itemize}
    \end{itemize}
\end{frame}

\begin{frame}
    \frametitle{Recommended Book(s)}
    \begin{columns}[t]
        \begin{column}[T]{5cm}
            \begin{itemize}
                \item \textbf{Introduction to Algorithms}
                    \begin{itemize}
                        \item Thomas Cormen,\\ Charles Leiserson,\\ Ronald Rivest,\\ Clifford Stein
                    \end{itemize}
                \item \textbf{Algorithms in C/C++/Java/}
                    \begin{itemize}
                        \item Robert Sedgewick
                    \end{itemize}
            \end{itemize}
        \end{column}
        \begin{column}[T]{5cm}
            \includegraphics[scale=0.5]{book.jpg}
        \end{column}
    \end{columns}
\end{frame}


\begin{frame}
    How about this: 90\% of programmers, given 2 hours, the high-level language of their choice (including pseudocode), and a description of binary search could not implement it correctly.

    \vspace{15mm}
    \pause
    Can you?
\end{frame}

\begin{frame}[fragile]
    \frametitle{Binary Search}
    \lstset{language=C,
        basicstyle=\footnotesize\ttfamily,
        numbers=left,
        numberstyle=\tiny\color{gray},
        numbersep=5pt,
        showspaces=false,
        showstringspaces=false,
        breaklines=true,
        breakatwhitespace=false,
        keywordstyle=\color{blue},
        commentstyle=\color{OliveGreen},
        stringstyle=\color{mauve},
        morekeywords={*,...},
    moredelim=**[is][\btHL]{`}{`}}
    \begin{lstlisting}
    int binary_search(int *array, int n, int x)
    {
        int lo = 0, hi = n - 1;
        while (lo < hi) {
            int mid = lo + (hi - lo) / 2;
            if (array[mid] < x)
                lo = mid + 1;
            else hi = mid;
        }

        if (lo == hi && array[lo] == x)
            return lo;
        return -1;
    }
    \end{lstlisting}
\end{frame}

\begin{frame}   
    \vspace*{\fill}
    \begingroup
    \centering
    \begin{center}
        \huge \textbf{Ad-hoc Problems}
    \end{center}
\endgroup
\vspace*{\fill}
\end{frame}

\begin{frame}
    \frametitle{Characteristics}
    \begin{itemize}
            \pause
        \item Most of them are very easy; some of them are very very hard
            \pause
        \item Don't require any special knowledge of algorithms
            \pause
        \item There is always at least one in competitions
    \end{itemize}
\end{frame}

\begin{frame}
    \frametitle{Strategies to solve}
    \begin{itemize}
            \pause
        \item Most of these problems are straightforward.
            \pause
        \item However, some ad-hoc problems require careful reading. 
            \pause
        \item Carefully sequencing the instructions given in the problem is usually enough to solve them.
            \pause
        \item Some require reasonable optimisations, and some degree of analysis to prune unnecessary steps.
            \pause
        \item If it's not obvious, then there's only one piece of advice I can give you:

            \vspace*{\fill}
            \begingroup
            \centering
            \pause
            \begin{center}
                \large \textbf{\textcolor{red}{Don't Panic!}}
            \end{center}
        \endgroup
        \vspace*{\fill}
\end{itemize}
\end{frame}

\begin{frame}
    \frametitle{Practice, practice, practice...}
    \begin{itemize}
            \pause
        \item To get really good at this, you need practice. \textbf{A lot} of practice.
            \pause
        \item So I've decided that every week I'll give you a set of \textbf{a lot} of problems so that you can practice.
            \pause
        \item Will usually be more than you can solve in a week, although I'd love to be proven wrong.
            \pause
        \item Solutions should be available one week after the problems were set.
    \end{itemize}

    \pause
    For this week, try your hand at these questions (all ad-hoc): \url{http://uva.onlinejudge.org/index.php?option=com_onlinejudge&Itemid=8&category=121}
\end{frame}

\begin{frame}
    \frametitle{More Info}
    \begin{itemize}
            \pause
        \item You can email me about anything related to this. I'll usually respond within a day or two.
            \pause
        \item All the slides, and everything we're going to do in these sessions, as well as solutions to weekly problems are on github:
            \begin{center}
                \href{https://github.com/sebastian-claici/acm\_sessions.git}{github.com/sebastian-claici/acm\_sessions.git}
            \end{center}
            \pause
    \end{itemize}
\end{frame}

\begin{frame}
    \vspace*{\fill}
    \begin{center}
        \huge \textbf{Thank You}
    \end{center}
    \vspace*{\fill}
\end{frame}

\end{document}
